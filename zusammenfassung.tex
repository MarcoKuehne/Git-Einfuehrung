\section{Zusammenfassung}
In diesem Bericht wurde auf die Grundlagen der Nutzung von Git eingegangen. Dabei wurde gezeigt, wie man lokale Repositories anlegen kann, wie die Arbeit lokal verfolgt wird und wie parallel an verschiedenen Versionen gearbeitet werden kann.

Zusätzlich wurde die Nutzung von sogenannten Remotes dargestellt, um das eigene Projekt zu sichern oder mit anderen Entwicklern (oder Nutzern) zu teilen. Diese Arbeit stellt keinerlei Anspruch an Vollständigkeit, sondern soll nur den Einstieg in die Arbeit mit Git ermöglichen.

Für weiterführende Informationen soll noch einmal das Buch \cite{ProGit} empfohlen werden, welches online zur freien Verfügung steht und alle hier aufgeführten Themen (und noch viele mehr) in tieferem Detail bespricht.